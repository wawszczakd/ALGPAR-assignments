\documentclass[12pt]{article}
\usepackage[utf8]{inputenc}
\usepackage[T1]{fontenc}
\usepackage{indentfirst}
\usepackage{amsmath}
\usepackage{amssymb}
\usepackage{natbib}
\usepackage{graphicx}
\usepackage{float}
\usepackage[a4paper, margin = 2 cm]{geometry}
\usepackage{fancyhdr}
\usepackage{wrapfig}
\usepackage{hyperref}
\usepackage{mathtools}
\usepackage{algorithm}
\usepackage{algpseudocode}
\usepackage{enumitem}
\usepackage{xcolor}

\title{Parameterized Algorithms assignment 3}
\author{Dominik Wawszczak}
\date{2025-01-23}

\begin{document}
	\setlength{\parindent}{0 cm}
	
	Dominik Wawszczak \hfill Parameterized Algorithms
	
	Student ID Number: 440014 \hfill Assignment 3
	
	Group Number: 1
	
	\bigskip
	\hrule
	\bigskip
	
	\textbf{Problem 1}
	
	\medskip
	
	We will show an fpt-reduction from the \textsc{Multicolored Clique} problem
	to the \textsc{Cut with Forbidden Pairs} problem. Let \((G, k)\), where
	\(V(G) = (V_{1}, V_{2}, \ldots, V_{k})\), be an	instance of the
	\textsc{Multicolored Clique} problem. For each \(i \in \{1, 2, \ldots,
	k\}\), let \(V_{i} = \{v_{i, 1}, v_{i, 2}, \ldots, v_{i, n_{i}}\}\) denote
	the set of vertices of color \(i\).
	
	\medskip
	
	We construct a directed graph \(D\) as follows:
	\begin{enumerate}
		\item Start with \(V(D) = \{s, t\}\) and \(E(D) = \emptyset\).
		\item For each \(i \in \{1, 2, \ldots, k\}\):
		      \begin{enumerate}[label=\theenumi.\arabic*]
		          \item Add \(n_{i} - 1\) vertices to \(V(D)\), denoted as
		                \(u_{i, 1}, u_{i, 2}, \ldots, u_{i, n_{i} - 1}\).
		          \item For every \(v_{i, j} \in V_{i}\), add an arc
		                \(a(v_{i, j})\) to \(E(D)\), where \(a(v_{i, j}) =
		                (u_{i, j - 1}, u_{i, j})\). Here, \(u_{i, 0} = s\) and
		                \(u_{i, n_{i}} = t\).
		      \end{enumerate}
	\end{enumerate}
	Let \(\mathcal{S}\) be the set of all pairs of arcs in \(D\) whose
	corresponding vertices are not adjacent in \(G\). Formally:
	\[ \mathcal{S} \ = \ \{\{a(v_{i_{1}, j_{1}}), a(v_{i_{2}, j_{2}})\} \ : \
	v_{i_{1}, j_{1}}, v_{i_{2}, j_{2}} \in V(G) \ \wedge \ \{v_{i_{1}, j_{1}},
	v_{i_{2}, j_{2}}\} \notin E(G)\} \text{.} \]
	The parameter \(k\) remains the same as in the original problem. Note that
	the size of the constructed instance is polynomial in the size of the
	original instance.
	
	\medskip
	
	We now prove the following equivalence:
	\[ (G, k) \in \textsc{Multicolored Clique} \ \iff \ (D, \mathcal{S}, k) \in
	\textsc{Cut with Forbidden Pairs} \text{.} \]
	
	First, assume there exists a multicolored clique \(C = \{v_{1, j_{1}},
	v_{2, j_{2}}, \ldots, v_{k, j_{k}}\}\) in \(G\). Define \(F =
	\{a(v_{1, j_{1}}), a(v_{2, j_{2}}), \ldots, a(v_{k, j_{k}})\}\). Since there
	are exactly \(k\) paths from \(s\) to \(t\) in \(D\), and
	\(a(v_{i, j_{i}})\) lies on the \(i\)-th path for each \(i\), \(F\)
	intersects every path from \(s\) to \(t\). Furthermore, for any \(\{a_{1},
	a_{2}\} \in \mathcal{S}\), \(|F \cap \{a_{1}, a_{2}\}| \leqslant 1\), as
	otherwise \(C\) would not be a clique in \(G\).
	
	\medskip
	
	Conversely, assume there exists a set \(F \subseteq E(D)\) of size \(k\)
	such that \(F\) intersects every path from \(s\) to \(t\), and \(|F \cap
	\{a_{1}, a_{2}\}| \leqslant 1\) for any \(\{a_{1}, a_{2}\} \in
	\mathcal{S}\). Let \(C = a^{-1}(F)\). Since \(F\) intersects every path from
	\(s\) to \(t\), \(C\) contains exactly one vertex from each \(V_{i}\).
	Additionally, for any two vertices \(v_{i_{1}, j_{i_{1}}},
	v_{i_{2}, j_{i_{2}}} \in C\), they are adjacent in \(G\), as otherwise
	\(\{a(v_{i_{1}, j_{i_{1}}}), a(v_{i_{2}, j_{i_{2}}})\}\) would belong to
	\(\mathcal{S}\). Hence, \(C\) is a multicolored clique in \(G\).
	
	\medskip
	
	From the above, we conclude that the \textsc{Cut with Forbidden Pairs}
	problem is \(\mathsf{W}[1]\)-hard when parameterized by \(k\).
	
	\medskip
	
	By the corollary 14.23 from the
	\href{https://parameterized-algorithms.mimuw.edu.pl/}
	{\textcolor{cyan}{Platypus Book}}, we know that there is no
	\(f(k) \cdot n^{o(k)}\)-time algorithm for the \textsc{Multicolored Clique}
	problem, for any computable function \(f\), assuming the Exponential Time
	Hypothesis (ETH). By the observation 14.22 from the same book, this implies
	that no \(f(k) \cdot n^{o(k)}\)-time algorithm exists for the \textsc{Cut
	with Forbidden Pairs} problem under ETH, which completes the proof.
\end{document}
