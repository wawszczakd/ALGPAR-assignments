\documentclass[12pt]{article}
\usepackage[utf8]{inputenc}
\usepackage[T1]{fontenc}
\usepackage{indentfirst}
\usepackage{amsmath}
\usepackage{amssymb}
\usepackage{natbib}
\usepackage{graphicx}
\usepackage{float}
\usepackage[a4paper, margin = 2 cm]{geometry}
\usepackage{fancyhdr}
\usepackage{wrapfig}
\usepackage{hyperref}
\usepackage{mathtools}
\usepackage{algorithm}
\usepackage{algpseudocode}
\usepackage{enumitem}
\usepackage{xcolor}

\title{Parameterized Algorithms assignment 3}
\author{Dominik Wawszczak}
\date{2025-01-23}

\begin{document}
	\setlength{\parindent}{0 cm}
	
	Dominik Wawszczak \hfill Parameterized Algorithms
	
	Student ID Number: 440014 \hfill Assignment 3
	
	Group Number: 1
	
	\bigskip
	\hrule
	\bigskip
	
	\textbf{Problem 1}
	
	\medskip
	
	We will show an fpt-reduction from the \textsc{Multicolored Clique} problem
	to the \textsc{Cut with Forbidden Pairs} problem. Let \((G, k)\), where
	\(V(G) = (V_{1}, V_{2}, \ldots, V_{k})\), be an	instance of the
	\textsc{Multicolored Clique} problem. For each \(i \in \{1, 2, \ldots,
	k\}\), let \(V_{i} = \{v_{i, 1}, v_{i, 2}, \ldots, v_{i, n_{i}}\}\) denote
	the set of vertices of color \(i\).
	
	\medskip
	
	We construct a directed graph \(D\) as follows:
	\begin{enumerate}
		\item Start with \(V(D) = \{s, t\}\) and \(E(D) = \emptyset\).
		\item For each \(i \in \{1, 2, \ldots, k\}\):
		      \begin{enumerate}[label=\theenumi.\arabic*]
		          \item Add \(n_{i} - 1\) vertices to \(V(D)\), denoted as
		                \(u_{i, 1}, u_{i, 2}, \ldots, u_{i, n_{i} - 1}\).
		          \item For every \(v_{i, j} \in V_{i}\), add an arc
		                \(a(v_{i, j})\) to \(E(D)\), where \(a(v_{i, j}) =
		                (u_{i, j - 1}, u_{i, j})\). Here, \(u_{i, 0} = s\) and
		                \(u_{i, n_{i}} = t\).
		      \end{enumerate}
	\end{enumerate}
	Let \(\mathcal{S}\) be the set of all pairs of arcs in \(D\) whose
	corresponding vertices are not adjacent in \(G\). Formally:
	\[ \mathcal{S} \ = \ \{\{a(v_{i_{1}, j_{1}}), a(v_{i_{2}, j_{2}})\} \ : \
	v_{i_{1}, j_{1}}, v_{i_{2}, j_{2}} \in V(G) \ \wedge \ \{v_{i_{1}, j_{1}},
	v_{i_{2}, j_{2}}\} \notin E(G)\} \text{.} \]
	The parameter \(k\) remains the same as in the original problem. Note that
	the size of the constructed instance is polynomial in the size of the
	original instance.
	
	\medskip
	
	We now prove the following equivalence:
	\[ (G, k) \in \textsc{Multicolored Clique} \ \iff \ (D, \mathcal{S}, k) \in
	\textsc{Cut with Forbidden Pairs} \text{.} \]
	
	First, assume there exists a multicolored clique \(C = \{v_{1, j_{1}},
	v_{2, j_{2}}, \ldots, v_{k, j_{k}}\}\) in \(G\). Define \(F =
	\{a(v_{1, j_{1}}), a(v_{2, j_{2}}), \ldots, a(v_{k, j_{k}})\}\). Since there
	are exactly \(k\) paths from \(s\) to \(t\) in \(D\), and
	\(a(v_{i, j_{i}})\) lies on the \(i\)-th path for each \(i\), \(F\)
	intersects every path from \(s\) to \(t\). Furthermore, for any \(\{a_{1},
	a_{2}\} \in \mathcal{S}\), \(|F \cap \{a_{1}, a_{2}\}| \leqslant 1\), as
	otherwise \(C\) would not be a clique in \(G\).
	
	\medskip
	
	Conversely, assume there exists a set \(F \subseteq E(D)\) of size \(k\)
	such that \(F\) intersects every path from \(s\) to \(t\), and \(|F \cap
	\{a_{1}, a_{2}\}| \leqslant 1\) for any \(\{a_{1}, a_{2}\} \in
	\mathcal{S}\). Let \(C = a^{-1}(F)\). Since \(F\) intersects every path from
	\(s\) to \(t\), \(C\) contains exactly one vertex from each \(V_{i}\).
	Additionally, for any two vertices \(v_{i_{1}, j_{i_{1}}},
	v_{i_{2}, j_{i_{2}}} \in C\), they are adjacent in \(G\), as otherwise
	\(\{a(v_{i_{1}, j_{i_{1}}}), a(v_{i_{2}, j_{i_{2}}})\}\) would belong to
	\(\mathcal{S}\). Hence, \(C\) is a multicolored clique in \(G\).
	
	\medskip
	
	From the above, we conclude that the \textsc{Cut with Forbidden Pairs}
	problem is \(\mathsf{W}[1]\)-hard when parameterized by \(k\).
	
	\medskip
	
	By the corollary 14.23 from the
	\href{https://parameterized-algorithms.mimuw.edu.pl/}
	{\textcolor{cyan}{Platypus Book}}, we know that there is no \(f(k) \cdot
	n^{o(k)}\)-time algorithm for the \textsc{Multicolored Clique} problem, for
	any computable function \(f\), assuming the Exponential Time Hypothesis
	(ETH). By the observation 14.22 from the same book, this implies that no
	\(f(k) \cdot n^{o(k)}\)-time algorithm exists for the \textsc{Cut with
	Forbidden Pairs} problem under ETH, which completes the proof.
	
	\newpage
	
	\textbf{Problem 2}
	
	\medskip
	
	Let \((D, s, t, k)\) be an instance of the \textsc{Double Cut} problem and
	let \(F\) be an optimal solution to this problem, meaning one with the
	smallest possible size. Denote \(P\) as the set of all the paths in \(D\)
	from \(s\) to \(t\). Define:
	\[ \quad X \ = \ \left\{a \in F \ : \ \underset{p \in P}{\forall} \ a \
	\text{is not the last arc on} \ p \ \text{that belongs to} \ F \right\}
	\text{.} \]
	and
	\[ Y \ = \ \left\{a \in F \ : \ \underset{p \in P}{\exists} \ a \
	\text{is the last arc on} \ p \ \text{that belongs to} \ F \right\} \]
	With these definitions, \(F = X \cup Y\) and \(X \cap Y = \emptyset\).
	
	\medskip
	
	\underline{Lemma 1} For any \(p \in P\), \(|p \cap X| \geqslant 1\).
	
	\medskip
	
	\underline{Proof of Lemma 1} Suppose, for the sake of contradiction, that
	there exists a path \(p \in P\) such that \(p \cap X = \emptyset\). Then,
	\(p \cap F = p \cap Y\). Let \(a\) be the first arc on \(p\) that belongs to
	\(Y\). Since \(a \in Y\), there exists another path \(p' \in P\) where \(a\)
	is the last arc belonging to \(F\). By merging the prefix of \(p\) ending
	with \(a\) with the suffix of \(p'\) starting after \(a\), we construct
	another path from \(s\) to \(t\) that contains only one arc from \(F\). This
	contradiction completes the proof of Lemma 1.
	
	\medskip
	
	\underline{Lemma 2} For some optimal solution \(F\), \(Y\) is an important
	\((s, t)\)-separator in \(D\).
	
	\medskip
	
	\underline{Proof of Lemma 2} If there exists a path \(p \in P\) that does
	not pass through \(Y\), then \(p \cap F = p \cap X\). Consequently, the last
	arc on \(p\) belonging to \(F\) would also be in \(X\), leading to a
	contradiction. Hence, \(Y\) is an \((s, t)\)-separator in \(D\).
	
	\smallskip
	
	Let \(R_{s}(Y)\) denote the set of vertices reachable from \(s\) in \(D
	\setminus Y\). Suppose there exists an \((s, t)\)-separator \(Y'\) such that
	\(|Y| \geqslant |Y'|\) and \(R_{s}(Y) \subseteq R_{s}(Y')\), where
	\(R_{s}(Y')\) is defined analogously. We claim that \(F' = X \cup Y'\) is
	also a valid solution.
	
	\smallskip
	
	If there exists a path \(p \in P\) containing at most one arc from \(F'\),
	that arc must belong to \(Y'\) because \(Y'\) is an \((s, t)\)-separator.
	By Lemma 1, this arc belongs to \(X\) as well. Since \(F\) is a solution,
	\(p\) also contains another arc \(a = (u, v) \in Y\), where \(u \in
	R_{s}(Y)\). This implies \(u \in R_{s}(Y')\), meaning there exists a path
	\(p'\) from \(s\) to \(u\) that avoids \(Y'\). By merging the prefix of
	\(p'\) ending at \(u\) with the suffix of \(p\) starting at \(u\), we obtain
	a path from \(s\) to \(t\) that contains no edges from \(Y'\). This
	contradicts the assumption that \(Y'\) is an \((s, t)\)-separator, proving
	that \(F'\) is another solution of size no greater than \(F\), completing
	the proof of Lemma 2.
	
	\medskip
	
	To find an optimal set \(X\) for a given \(Y\), we can merge all vertices
	with outgoing arcs in \(Y\) into a single vertex and then run a maximum flow
	algorithm from \(s\) to this new vertex. The correctness of this approach
	follows from Lemma 1, and its time complexity is polynomial in \(n\), where
	\(n\) is the number of vertices in \(D\).
	
	\medskip
	
	The final algorithm proceeds as follows:
	\begin{enumerate}
		\item Iterate over all possible important separators \(Y\) of size at
		      most \(k\).
		\item For each \(Y\), compute the optimal set \(X\).
		\item If \(|X| + |Y| \leqslant k\), return \texttt{true}.
		\item If no solution is found, return \texttt{false}.
	\end{enumerate}
	The time complexity of this algorithm is \(4^{k} \cdot n^{\mathcal{O}(1)}\),
	as important separators can be enumerated in time \(4^{k} \cdot
	n^{\mathcal{O}(1)}\). This is fixed-parameter tractable when parameterized
	by \(k\), concluding the proof.
	
	\newpage
	
	\textbf{Problem 3}
	
	\medskip
	
	We will use the representative sets technique. Define
	\begin{align*}
		\mathcal{P}_{u \to v}^{p} \ = \ \big\{ V(P) \ : \ &P \
		\text{consists of} \ \big\lfloor \tfrac{p - 1}{5} \big\rfloor \
		\text{vertex-disjoint cycles of length} \ 5 \\
		&\text{and a path of length} \ (p - 1) \ \text{mod} \ 5 \ \text{from} \
		u \ \text{to} \ v \big\} \text{,}
	\end{align*}
	for every \(u, v \in V(G)\). Let \(\widehat{\mathcal{P}}_{u \to v}^{p}
	\subseteq_{\text{rep}}^{5k - p} \mathcal{P}_{u \to v}^{p}\). We will
	calculate \(\widehat{\mathcal{P}}_{u \to v}^{p}\) for each \(p \in \{1, 2,
	\ldots, 5k\}\), maintaining the invariant \(\big|
	\widehat{\mathcal{P}}_{u \to v}^{p} \big| \leqslant \binom{5k}{p}\).
	
	\medskip
	
	For \(p = 1\), we have \(\widehat{\mathcal{P}}_{u \to u}^{1} = \{\{u\}\}\),
	for every \(u \in V(G)\). Assume that
	\(\widehat{\mathcal{P}}_{u \to v}^{p - 1}\) has already been computed. If
	\((p - 1) \ \text{mod} \ 5 \neq 0\), let
	\[ \widetilde{\mathcal{P}}_{u \to v}^{p} \ = \
	\bigcup\limits_{\{w, v\} \in E(G)} \Big\{ P \cup \{v\} \ : \ P \in
	\widehat{\mathcal{P}}_{u \to w}^{p - 1} \ \wedge \ v \notin P \Big\}
	\text{,} \]
	for every \(u, v \in V(G)\). Otherwise, let
	\[ \widetilde{\mathcal{P}}_{u \to u}^{p} \ = \
	\bigcup\limits_{\{v, w\} \in E(G)} \Big\{ P \cup \{u\} \ : \ P \in
	\widehat{\mathcal{P}}_{v \to w}^{p - 1} \ \wedge \ u \notin P \Big\}
	\text{,} \]
	for every \(u \in V(G)\). Here we only consider vertex pairs connected by an
	edge to close the cycle. Finally, let
	\[ \widehat{\mathcal{P}}_{u \to v}^{p} \ = \ \text{trim} \Big(
	\widetilde{\mathcal{P}}_{u \to v}^{p} \Big) \text{,} \]
	for every \(u, v \in V(G)\). The trim operation applies the FLS (Fomin,
	Lokshtanov, Saurabh) algorithm, ensuring the invariant is maintained.
	
	\medskip
	
	The final answer is
	\[ \underset{\{u, v\} \in E(G)}{\exists} \
	\widehat{\mathcal{P}}_{u \to v}^{5k} \ \neq \ \emptyset \text{,} \]
	since the last cycle must be closed by an edge.
	
	\medskip
	
	The size of \(\widetilde{\mathcal{P}}_{u \to v}^{p}\) is at most \(n^{2}
	\cdot \binom{5k}{p}\) due to the invariant. Therefore, the time needed to
	compute \(\widehat{\mathcal{P}}_{u \to v}^{5k}\) is bounded by
	\((2^{5 \omega})^{k} \cdot n^{\mathcal{O}(1)}\), where \(\omega\) is the
	matrix multiplication constant. Given that there are \(5k\) iterations, each
	requiring this step \(n^{2}\) times, the total time complexity is
	\((2^{5 \omega})^{k} \cdot n^{\mathcal{O}(1)}\). This completes the proof,
	as \(2^{\omega} \approx 2^{2.371552} \approx 5.174975\), while \(2e \approx
	5.436564\).
	
	\bigskip
	
	\textbf{Problem 4}
	
	\medskip
	
	Without loss of generality, assume the decomposition is nice. For any
	subtree \(H\) of the decomposition and its boundary \(\partial H\), let \(f
	: V(\partial H) \to \{\text{blue}, \text{red}\}\). Define
	\(\text{dp}_{H}(f)\) as the maximum number of edges in \(H\) with endpoints
	of different colors, for some coloring \(g : V(H) \to \{\text{blue},
	\text{red}\}\) such that for every \(u \in V(\partial H)\), \(g(u) = f(u)\).
	
	\medskip
	
	To compute \(\text{dp}_{H}\), we consider the following cases:
	\begin{enumerate}
		\item \(H = \emptyset\)
		      
		      In this case, \(\text{dp}_{H}(\text{empty function}) = 0\).
		
		\item \(H = \text{introduceVertex}(H', u)\)
		      
		      Here, \(\text{dp}_{H}(f) = \text{dp}_{H'}(f')\), where \(f =
		      f'[u \to \text{color}]\), and color is either blue or red. We use
		      the notation \(g[p \to q]\) to denote the function h defined by:
		      \[ h(x) \ = \ \begin{cases} q \text{,} & \text{if} \ x = p
		      \text{,} \\
		      g(x) \text{,} & \text{otherwise.} \end{cases} \]
		
		\item \(H = \text{introduceEdge}(H', u, v)\)
		      
		      In this case, \(\text{dp}_{H}(f) = \text{dp}_{H'}(f) + [f(u) \neq
		      f(v)]\), where \([P]\) denotes the Iverson bracket, i.e., \([P] =
		      1\) if \(P\) is true and \([P] = 0\) otherwise.
		
		\item \(H = \text{forgetVertex}(H', u)\)
		      
		      Here, \(\text{dp}_{H}(f) = \max(\{
		      \text{dp}_{H'}(f[u \to \text{blue}]),
		      \text{dp}_{H'}(f[u \to \text{red}])\})\).
		
		\item \(H = \text{merge}(H', H'')\)
		      
		      In this case \(\partial H = \partial H' = \partial H''\), and we
		      set
		      \[ \text{dp}_{H}(f) = \text{dp}_{H'}(f) + \text{dp}_{H''}(f) -
		      |\{\{u, v\} \ : \ \{u, v\} \in E(\partial H) \ \wedge \ f(u) \neq
		      f(v)\}| \text{.} \]
		      We subtract the number of edges in \(\partial H\) with endpoints
		      of different colors to avoid double counting.
	\end{enumerate}
	The answer to the \textsc{Max Cut} problem is
	\(\text{dp}_{T}(\text{empty function})\), where \(T\) is the full
	decomposition. The time complexity of this algorithm is \(2^{t} \cdot
	n^{\mathcal{O}(1)}\).
	
	\medskip
	
	Now, we prove that under the Strong Exponential Time Hypothesis (SETH), no
	algorithm can run in time \(2^{t - \varepsilon} \cdot n^{\mathcal{O}(1)}\)
	for any \(\varepsilon > 0\). Given an instance \(\varphi\) of the SAT
	problem with \(n\) variables, we construct a graph \(G\) as follows:
	\begin{enumerate}
		\item Create a special vertex \(r\).
		\item For each variable \(x_{i}\), add a vertex \(u_{i}\) to \(G\).
		      Define \(u(l)\) as the vertex corresponding to literal \(l\).
		\item For each clause \(c \in \varphi\), let \(c = (l_{c, 1} \ \vee \
		      l_{c, 2} \ \vee \ \ldots \ \vee \ l_{c, |c|})\), where \(l_{c, j}
		      \in \bigcup\limits_{i \in \{1, 2, \ldots, n\}} \{x_{i}, \neg
		      x_{i}\}\), for each \(j \in \{1, \ldots, |c|\}\). Without loss of
		      generality, assume \(|c| \leqslant n\), as otherwise, some two
		      literals would correspond to the same variable. Additionally,
		      assume the first \(k_{c}\) literals correspond to normal
		      variables, while the remaining \(|c| - k_{c}\) are negations.
		      \begin{enumerate}[label=\theenumi.\arabic*]
		          \item Create \(2n + 2\) vertices: \(v_{c, 1}, v_{c, 2},
		                \ldots, v_{c, 2n + 2}\). Arrange them in a cycle
		                \(C_{c} = (r, v_{c, 1}, v_{c, 2}, \allowbreak \ldots,
		                v_{c, 2n + 2})\), adding \(2n + 1\) edges between every
		                pair of consecutive vertices. This results in a
		                non-simple graph, but we will later address this issue.
		          \item For each \(j \in \{1, 2, \ldots, k_{c}\}\), add edges
		                \(\{v_{c, 2j - 1}, u(l_{c, j})\}\) and
		                \(\{v_{c, 2j}, u(l_{c, j})\}\).
		          \item For each \(j \in \{k_{c} + 1, k_{c} + 2, \ldots,
		                |c|\}\), add edges \(\{v_{c, 2j}, u(l_{c, j})\}\) and
		                \(\{v_{c, 2j + 1}, u(l_{c, j})\}\).
		      \end{enumerate}
		      Note that by this construction, every vertex in \(C_{c}\), except
		      for \(r\), has at most one neighbor outside the cycle.
	\end{enumerate}
	Let
	\[ M = ((2n + 1) \cdot (n + 1) + 1) \cdot m + \sum\limits_{c \in \varphi}
	|c| \text{,} \]
	where \(m\) is the number of clauses in \(\varphi\). We claim that
	\(\varphi\) is satisfiable if and only if the max cut in \(G\) is at least
	\(M\).
	
	\medskip
	
	First, assume that \(\varphi\) is satisfiable. We assign \(r\) the color
	blue. For each variable \(x_{i}\), we color \(u_{i}\) blue if \(x_{i}\) is
	true, and red otherwise. For every clause \(c\), let \(l_{c, j}\) be the
	first true literal in \(c\). Let \(v_{c, x}\) and \(v_{c, x + 1}\) be the
	two neighbors of \(u(l_{c, j})\) in \(C_{c}\). For all \(y \in \{1, 2,
	\ldots, x\}\), we color \(v_{c, y}\) blue if \(y\) is even, and red
	otherwise. For all \(y \in \{x + 1, x + 2, \ldots, 2n + 2\}\), we color
	\(v_{c, y}\) red if \(y\) is even, and blue otherwise. Note that
	\(v_{c, x}\) and \(v_{c, x + 1}\) share the same color, which differs from
	the color of \(u(l_{c, j})\), while the remaining vertices in \(C_{c}\)
	alternate in color. Thus, the number of edges with endpoints of different
	colors in \(C_{c}\) is exactly \((2n + 1) \cdot (n + 1) + 1 + |c|\), since
	the neighbors of every literal vertex other than \(u(l_{c, j})\) have
	different colors. Therefore, the total number of edges with endpoints of
	different colors is \(M\).
	
	\medskip
	
	Conversely, assume that the max cut in \(G\) is at least \(M\). Without loss
	of generality, assume the color of \(r\) is blue. Since for every clause \(c
	\in \varphi\), the cycle \(C_{c}\) has length \(2n + 3\), there can be at
	most \(n + 1\) pairs of consecutive vertices with different colors. If there
	exists a clause \(c \in \varphi\) with fewer than \(n + 1\) such pairs, the
	number of edges with endpoints of different colors in \(C_{c}\) would be at
	most \((2n + 1) \cdot n + 2|c|\), which is strictly less than \((2n + 1)
	\cdot (n + 1)\). Therefore, for each \(c \in \varphi\), there must be
	exactly one pair of consecutive vertices in \(C_{c}\) with the same color.
	Moreover, for any clause \(c \in \varphi\), the contribution of \(C_{c}\) to
	the cut is at most \((2n + 1) \cdot (n + 1) + 1 + |c|\), implying that it is
	exactly \((2n + 1) \cdot (n + 1) + 1 + |c|\). Thus, there exists a literal
	\(l_{c, j}\) such that \(u(l_{c, j})\) has a different color from both its
	neighbors in \(C_{c}\). This is the literal that satisfies the clause \(c\),
	concluding the proof of the equivalence.
	
	\medskip
	
	To remove multi-edges, replace each with a path of length \(3\). It is easy
	to verify that the equivalence still holds.
	
	\medskip
	
	Since the size of \(G\) is polynomial in the size of \(\varphi\), it remains
	to calculate the bound on the treewidth of \(G\). The treewidth of each path
	of length \(3\) is \(1\). We can place all the decompositions of these paths
	next to each other, then insert the vertices from \(C_{c}\) that each path
	connects, resulting in a tree decomposition of treewidth 3. Next, we add
	\(r\) and the vertices corresponding to the variables to each bag, which
	gives a tree decomposition of \(G\) with treewidth \(n + 4\).
	
	\medskip
	
	Therefore, an algorithm solving \textsc{Max Cut} in time
	\(2^{t - \varepsilon} \cdot n^{\mathcal{O}(1)}\) for some \(\varepsilon >
	0\) contradicts SETH, which completes the proof.
\end{document}
