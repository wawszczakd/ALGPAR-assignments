\documentclass[12pt]{article}
\usepackage[utf8]{inputenc}
\usepackage[T1]{fontenc}
\usepackage{indentfirst}
\usepackage{amsmath}
\usepackage{amssymb}
\usepackage{natbib}
\usepackage{graphicx}
\usepackage{float}
\usepackage[a4paper, margin = 2 cm]{geometry}
\usepackage{fancyhdr}
\usepackage{wrapfig}
\usepackage{hyperref}
\usepackage{mathtools}
\usepackage{algorithm}
\usepackage{algpseudocode}

\title{Parameterized Algorithms assignment 2}
\author{Dominik Wawszczak}
\date{2024-12-01}

\begin{document}
	\setlength{\parindent}{0 cm}
	
	Dominik Wawszczak \hfill Parameterized Algorithms
	
	Student ID Number: 440014 \hfill Assignment 2
	
	Group Number: 1
	
	\bigskip
	\hrule
	\bigskip
	
	\textbf{Problem 1}
	
	\medskip
	
	If \(G\) is not connected, we can analyze each connected component
	independently, subtracting the combined sizes of all other components from
	\(k\). Thus, for the remainder of the solution, we assume that \(G\) is
	connected.
	
	\medskip
	
	
	If there exists a subset of vertices \(X\) such that \(|X| \leqslant k\) and
	\(G \setminus X\) is a tree, then \(\text{tw}(G) \leqslant k + 1\), as
	\(\text{tw}(G \setminus X) = 1\), where \(\text{tw}\) denotes treewidth. We
	will use the algorithm presented in the lecture, which runs in time \(27^{l}
	\cdot l^{\mathcal{O}(1)} \cdot n^{2}\), where \(l\) is the target treewidth
	and \(n\) is the number of vertices. We apply it to \(G\) with \(l = k +
	1\). The algorithm will yield one of two possible outcomes:
	\begin{enumerate}
		\item Confirmation that \(\text{tw}(G) > k + 1\).
		      
		      In this case, we conclude that no such subset \(X\) exists.
		
		\item A tree decomposition of width at most \(4k + 8\).
		      
		      The remainder of the solution focuses on this scenario.
	\end{enumerate}
	
	\medskip
	
	We now proceed with dynamic programming, assuming the decomposition is nice.
	For any subtree \(H\) of the decomposition and its boundary \(\partial H\),
	let \(f : V(\partial H) \to \{0, \ldots, 4k + 9\}\). Define
	\(\text{dp}_{H}(f)\) as the size of the minimum set \(Y \subseteq
	V(H \setminus \partial H)\) satisfying the following conditions:
	\begin{enumerate}
		\item \(H \setminus \big( Y \cup f^{-1}(0) \big)\) is a forest,
		\item for all \(u, v \in V(\partial H) \setminus f^{-1}(0)\), \(u\) and
		\(v\) are in the same connected component of \(H \setminus \big( Y \cup
		f^{-1}(0) \big)\) if and only if \(f(u) = f(v)\).
	\end{enumerate}
	If no such set \(Y\) exists for a given \(f\), we define \(\text{dp}_{H}(f)
	= \infty\).
	
	\medskip
	
	To compute \(\text{dp}_{H}\), we consider the following cases:
	\begin{enumerate}
		\item \(H = \text{introduceVertex}(H', u)\)
		      
		      Here, \(\text{dp}_{H}(f) = \min(\{\text{dp}_{H'}(f') \ : \ f =
		      f'[u \mapsto f(u)] \ \wedge \ f'^{-1}(f(u)) = \emptyset\})\),
		      where \(\min(\emptyset) = \infty\). We use the notation
		      \(f[a \mapsto b]\) to denote the function \(g\) defined by:
		      \[ g(x) \ = \ \begin{cases} b \text{,} & \text{if} \ x = a
		      \text{,} \\ f(x) \text{,} & \text{otherwise.} \end{cases} \]
		
		\item \(H = \text{introduceEdge}(H', u, v)\)
		      
		      In this case, \(\text{dp}_{H}(f) = \min(\{\text{dp}_{H'}(f') \ : \
		      f'(u) \neq f'(v) \ \wedge \ (f'(w) = f'(u) \ \vee \ f'(w) = f'(v))
		      \ \Rightarrow \ f(w) = f(u) = f(v)\})\).
		
		\item \(H = \text{forgetVertex}(H', u)\)
		      
		      Here, \(\text{dp}_{H}(f) = \min(\{\text{dp}_{H'}(f') + [k = 0] \ :
		      k \in \{0, \ldots, 4k + 9\} \ \wedge \ f' = f[u \mapsto k]\})\),
		      where \([P]\) denotes the Iverson bracket, i.e., \([P] = 1\) if
		      \(P\) is true, and \([P] = 0\) otherwise.
		
		\item \(H = \text{merge}(H', H'')\)
		      
		      In this case \(\partial H' = \partial H''\), and we calculate
		      \(\text{dp}_{H}\) as \(\text{dp}_{H}(f) = \text{dp}_{H'}(f) +
		      dp_{H''}(f)\).
	\end{enumerate}
	The answer is \(\text{dp}_{T}(\text{empty function}) \leqslant k\), where
	\(T\) is the entire decomposition.
	
	\medskip
	
	The total complexity is bounded by
	\[ 27^{k + 1} \cdot (k + 1)^{\mathcal{O}(1)} \cdot n^{2} +
	n^{\mathcal{O}(1)} \cdot \left( (4k + 10)^{4k + 10} \right)^{2} \text{,} \]
	thus this algorithm is FPT when parameterized by \(k\).
\end{document}
