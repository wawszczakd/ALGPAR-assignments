\documentclass[12pt]{article}
\usepackage[utf8]{inputenc}
\usepackage[T1]{fontenc}
\usepackage{indentfirst}
\usepackage{amsmath}
\usepackage{amssymb}
\usepackage{natbib}
\usepackage{graphicx}
\usepackage{float}
\usepackage[a4paper, margin = 2 cm]{geometry}
\usepackage{fancyhdr}
\usepackage{wrapfig}
\usepackage{hyperref}
\usepackage{mathtools}
\usepackage{algorithm}
\usepackage{algpseudocode}

\title{Parameterized Algorithms assignment 1}
\author{Dominik Wawszczak}
\date{2024-10-29}

\begin{document}
	\setlength{\parindent}{0 cm}
	
	Dominik Wawszczak \hfill Parameterized Algorithms
	
	student id number: 440014 \hfill assignment 1
	
	group number: 1
	
	\bigskip
	\hrule
	\bigskip
	
	\textbf{Problem 1}
	
	\medskip
	
	Let
	\[ V(G) \ = \ \{v_{1}, \ldots, v_{n}\} \quad \text{and} \quad S_{n} \ = \
	\left\{ \sigma \ : \ \sigma \in \{1, \ldots, n\}^{\{1, \ldots, n\}} \ \wedge
	\ \sigma \ \text{is a bijection} \right\} \text{.} \]
	Define
	\[ \text{cutwidth}_{\sigma}(i) \ = \ |\{(u, v) \ : \ u \in \{v_{\sigma_{1}}
	\ldots v_{\sigma_{i}}\} \ \wedge \ v \in \{v_{\sigma_{i + 1}}, \ldots,
	v_{\sigma_{n}}\} \ \wedge \ (u, v) \in E(G)\}| \text{,} \]
	where \(\sigma \in S_{n}\) is any permutation. The objective is to find a
	permutation \(\sigma \in S_{n}\) that minimizes the value
	\[ \underset{i \in \{1, \ldots, n - 1\}}{\max} \
	\text{cutwidth}_{\sigma}(i) \]
	and to calculate this minimum.
	
	\medskip
	
	Define the function
	\[ \text{out}(X) \ = \ |\{(u, v) \ : \ u \in X \ \wedge \ v \in V(G)
	\setminus X \ \wedge \ (u, v) \in E(G)\}| \text{,} \]
	where \(X\) is any subset of \(V(G)\). Then
	\[ \text{cutwidth}_{\sigma}(i) \ = \ \text{out}(\{v_{\sigma_{1}}, \ldots,
	v_{\sigma_{i}}\}) \text{.} \]
	For any given \(X\), \(\text{out}(X)\) can be calculated in time
	\(n^{\mathcal{O}(1)}\).
	
	\medskip
	
	We will use dynamic programming over subsets. Define
	\[ \text{dp}(X) \ = \ \min \left\{ \underset{i \in
	\{1, \ldots, |X| - 1\}}{\max} \ \text{cutwidth}_{\sigma}(i) \ : \ \sigma \in
	S_{n} \ \wedge \ \{v_{\sigma_{1}}, \ldots, v_{\sigma_{|X|}}\} = X \right\}
	\text{,} \]
	where \(X\) is any subset of \(V(G)\). Then
	\begin{align*}
		\text{dp}(\emptyset) \ &= \ 0 \text{,} \\
		\text{dp}(X) \ &= \ \min \{\max(\text{dp}(X \setminus \{x\}),
		\text{out}(X \setminus \{x\})) \ : \ x \in X\} \text{.}
	\end{align*}
	The answer to the problem is \(\text{dp}(\{1, \ldots, n\})\). To compute
	\(\text{dp}\) for all subsets of \(V(G)\), we iterate over subsets in
	non-decreasing order of size. The time complexity of this algorithm is
	\(2^{n} \cdot n^{\mathcal{O}(1)}\).
	
	\bigskip
	
	\textbf{Problem 2}
	
	\medskip
	
	\underline{Lemma 1} A set of points \(S\), in which no three points are
	collinear, does not form the vertices of a convex polygon if and only if
	there exist points \(A, B, C, D \in S\) such that \(D\) lies inside
	\(\triangle ABC\).
	
	\medskip
	
	\underline{Proof of lemma 1} The implication ,,to the left'' is obvious,
	so we only need to prove the implication ,,to the right''. Let \(\{H_{1},
	\ldots, H_{h}\}\) be the convex hull of the set \(S\), with the assumption
	that these points are ordered counterclockwise along the hull. Let \(D\) be
	any point in \(S\) that does not belong to the hull, and let \(A = H_{1}\).
	There exists exactly one \(i \in \{2, \ldots, h - 1\}\) such that points
	\(H_{2}, \ldots, H_{i}\) lie on one side of the line \(AD\), while points
	\(H_{i + 1}, \ldots, H_{h}\) lie on the other side. Taking \(B = H_{i}\) and
	\(C = H_{i + 1}\), we will obtain the desired points.
	
	\newpage
	
	According to lemma 1, if there exist points \(A, B, C, D \in S\) such that
	\(D\) lies inside \(\triangle ABC\), then at least of these points must be
	removed from \(S\). This observation leads to the following algorithm:
	
	\begin{algorithm}
		\caption{ConvexDeletion} \label{alg:alg1}
		\begin{algorithmic}[1]
			\Procedure{ConvexDeletion}{$S, k$}
			    \If{no four points $A, B, C, D \in S$ satisfy that $D$ lies
			        inside $\triangle ABC$}
			        \State \Return true
			    \EndIf
			    \If{$k \leqslant 0$}
			        \State \Return false
			    \EndIf
			    \State Choose points $(A, B, C, D) \in S$ such that $D$ lies
			           inside $\triangle ABC$
			    \State \Return \Call{ConvexDeletion}{$S \setminus \{A\}, k - 1$}
			           \textbf{or}
			           \Call{ConvexDeletion}{$S \setminus \{B\}, k - 1$}
			           \textbf{or}
			    \Statex \hspace{4.685em}
			           \Call{ConvexDeletion}{$S \setminus \{C\}, k - 1$}
			           \textbf{or}
			           \Call{ConvexDeletion}{$S \setminus \{D\}, k - 1$}
			\EndProcedure
		\end{algorithmic}
	\end{algorithm}
	
	Finding such quadruples of points \(A, B, C, D\) can easily be done in
	\(\mathcal{O} \big( n^{4} \big)\) time by examining all quadruples of
	points, calculating the relevant cross products for each, and comparing
	their signs. This can also be done in \(\mathcal{O} (n \log n)\) time by
	computing the convex hull using Graham's algorithm and applying the
	constructive proof of lemma 1.
	
	\medskip
	
	The depth of the recursion tree fro of the algorithm \ref{alg:alg1} is at
	most \(k\), since with each recursive call, the parameter \(k\) decreases
	by \(1\). Each node of this tree has at most four children, which gives us
	an upper bound on the number of nodes in the tree:
	\[ \sum\limits_{i = 0}^{k} 4^{i} \ = \ \frac{4^{k + 1} - 1}{3} \ = \
	\mathcal{O} \big( 4^{k} \big) \text{.} \]
	Therefore, the overall complexity of the algorithm \ref{alg:alg1} is
	\(\mathcal{O} \big( 4^{k} \big) \cdot n^{\mathcal{O}(1)}\).
	
	\bigskip
	
	\textbf{Problem 3}
	
	\medskip
	
	Let \(d = 10\). We will begin by reducing the size of \(\mathcal{F}\) to a
	polynomial in \(k\).
	
	\medskip
	
	If \(|\mathcal{F}| \leqslant k^{d + 1}\), no reduction is necessary.
	Otherwise, there exists an element \(a_{1} \in \bigcup \mathcal{F}\) such
	that the subset \(\mathcal{A}_{1} = \{A \in \mathcal{F} \ : \ a_{1} \in
	A\}\) has size at least \(k^{d} + 1\). If this were not the case, a hitting
	set of size \(k\) would cover at most \(k^{d + 1}\) sets. We either include
	\(a_{1}\) in the hitting set or exclude it. If we exclude it, then there
	exists an element \(a_{2} \in \bigcup \mathcal{F} \setminus \{a_{1}\}\) such
	that \(\mathcal{A}_{2} = \{A \in \mathcal{A}_{1} \ : \ a_{2} \in A\}\) has
	size at least \(k^{d - 1} + 1\), and so on.
	
	\medskip
	
	By repeating this process until it is possible, we obtain a set of \(l\)
	distinct elements \(\{a_{1}, \ldots, a_{l}\}\) and a corresponding set of
	families \(\{\mathcal{A}_{1}, \ldots, \mathcal{A}_{l}\}\) such that for each
	\(i \in \{1, \ldots, l\}\), \(\mathcal{A}_{i - 1} \supseteq
	\mathcal{A}_{i}\) (with \(\mathcal{A}_{0} = \mathcal{F})\),
	\(|\mathcal{A}_{i}| \geqslant k^{d + 1 - i} + 1\), and
	\[ \underset{j \in \{i, i + 1, \ldots, l\}}{\forall} \
	\underset{A \in \mathcal{A}_{j}}{\forall} \ a_{i} \in A \text{.} \]
	
	Suppose, for the sake of contradiction, that \(l > d\). Then
	\(|\mathcal{A}_{d + 1}| \geqslant 2\), and all sets in
	\(\mathcal{A}_{d + 1}\) would contain \(\{a_{1}, \ldots, a_{d + 1}\}\),
	which contradicts the assumption that for any two distinct sets \(A, B \in
	\mathcal{F}\), \(|A \cap B| \leqslant d\).
	
	\medskip
	
	At least one element from \(\{a_{1}, \ldots, a_{l}\}\) must be included in
	the hitting set. This can be verified, as otherwise, to cover the entire
	family \(\mathcal{A}_{l}\) (which has size at least \(k^{d + 1 - l} + 1\)),
	there would need to be an element \(a_{l + 1}\) contained in at least
	\(k^{d - l} + 1\) sets from \(\mathcal{A}_{l}\). This would contradict the
	fact that the process was repeated until it was possible.
	
	\medskip
	
	If we include any \(a \in \{a_{1}, \ldots, a_{l}\}\) in the hitting set,
	every set \(A \in \mathcal{A}_{1}\) will be covered. Therefore, we can apply
	the following reduction, until it is possible:
	\begin{itemize}
		\item[R1:] If \(|\mathcal{F}| > k^{d + 1}\), find \(\{a_{1}, \ldots,
		           a_{l}\}\) and \(\{\mathcal{A}_{1}, \ldots,
		           \mathcal{A}_{l}\}\), then replace \(\mathcal{F}\) with
		           \((\mathcal{F} \setminus \mathcal{A}_{1}) \cup \{\{a_{1},
		           \ldots, a_{l}\}\}\).
	\end{itemize}
	Note that after applying R1, the condition that for any two distinct sets
	\(A, B \in \mathcal{F}\), \(|A \cap B| \leqslant d\) still holds, as the
	newly added set is a subset of one or more sets that were originally in
	\(\mathcal{F}\).
	
	\medskip
	
	The sets \(\{a_{1}, \ldots, a_{l}\}\) and \(\{\mathcal{A}_{1}, \ldots,
	\mathcal{A}_{l}\}\) can be found in polynomial time with respect to the
	input size. To identify \(a_{i}\), we iterate through each \(a \in \bigcup
	\mathcal{A}_{i - 1}\), checking which sets in \(\mathcal{A}_{i - 1}\)
	contain it.
	
	\medskip
	
	Applying R1 will require polynomial time overall, as each application of R1
	decreases the size of \(\mathcal{F}\) by at least \(k^{d} + 1 - 1 = k^{d} >
	0\), so R1 will be applied at most \(|\mathcal{F}|\) times.
	
	\medskip
	
	Once the size of \(|\mathcal{F}|\) has been reduced to
	\(\mathcal{O}\big(k^{d + 1}\big)\), we still need to reduce the size of
	\(\bigcup \mathcal{F}\). We apply the following reduction as long as it is
	possible:
	\begin{itemize}
		\item[R2:] If there exists a set \(A \in \mathcal{F}\) such that for
		           every \(B \in \mathcal{F} \setminus \{A\}\), \(A \cap B =
		           \emptyset\), we replace \(\mathcal{F}\) with \(\mathcal{F}
		           \setminus \{A\}\), \(k\) with \(k - 1\), and add any \(a \in
		           A\) to the hitting set, unless \(A\) is empty, in which case
		           we reject.
	\end{itemize}
	Now we assume that for any two distinct sets \(A, B \in \mathcal{F}\), \(A
	\cap B \neq \emptyset\), which enables us to apply the next reduction until
	it is no longer possible:
	\begin{itemize}
		\item[R3:] If there exists an element \(a \in \bigcup \mathcal{F}\) that
		           is contained in only one set in \(\mathcal{F}\), replace
		           \(\mathcal{F}\) with \(\{A \setminus \{a\} \ : \ A \in
		           \mathcal{F}\}\).
	\end{itemize}
	This reduction is valid, as if \(a\) were required in the hitting set, it
	could be replaced with any other element from the set in \(\mathcal{F}\)
	that contains it, particularly an element included in at least two sets in
	\(\mathcal{F}\).
	
	\medskip
	
	After applying R2 and R3, every element in \(\bigcup \mathcal{F}\) is
	contained in at least two sets in \(\mathcal{F}\). Therefore, the size of
	\(\bigcup \mathcal{F}\) is bounded by:
	\[ \left| \bigcup \mathcal{F} \right| \ \leqslant \ \frac{1}{2} \cdot
	\sum\limits_{A, B \in \mathcal{F} \ \wedge \ A \neq B} |A \cap B| \
	\leqslant \ \frac{k^{d + 1} \big(k^{d + 1} - 1\big)d}{2} \ = \
	k^{\mathcal{O}(1)} \text{,} \]
	which completes the proof.
\end{document}
