\documentclass[12pt]{article}
\usepackage[utf8]{inputenc}
\usepackage[T1]{fontenc}
\usepackage{indentfirst}
\usepackage{amsmath}
\usepackage{amssymb}
\usepackage{natbib}
\usepackage{graphicx}
\usepackage{float}
\usepackage[a4paper, margin = 2 cm]{geometry}
\usepackage{fancyhdr}
\usepackage{wrapfig}
\usepackage{hyperref}
\usepackage{mathtools}
\usepackage{algorithm}
\usepackage{algpseudocode}

\title{Algorytmy parametryzowane praca domowa 1}
\author{Dominik Wawszczak}
\date{2024-10-29}

\begin{document}
	\setlength{\parindent}{0 cm}
	
	Dominik Wawszczak \hfill Algorytmy Parametryzowane
	
	numer indeksu: 440014 \hfill praca domowa 1
	
	numer grupy: 1
	
	\bigskip
	\hrule
	\bigskip
	
	\textbf{Zadanie 1}
	
	\medskip
	
	Oznaczmy
	\[ V(G) \ = \ \{v_{1}, \ldots, v_{n}\} \quad \text{oraz} \quad S_{n} \ = \
	\left\{ \sigma \ : \ \sigma \in \{1, \ldots, n\}^{\{1, \ldots, n\}} \ \wedge
	\ \sigma \ \text{jest bijekcją} \right\} \text{.} \]
	Niech
	\[ \text{cutwidth}_{\sigma}(i) \ = \ |\{(u, v) \ : \ u \in \{v_{\sigma_{1}}
	\ldots v_{\sigma_{i}}\} \ \wedge \ v \in \{v_{\sigma_{i + 1}}, \ldots,
	v_{\sigma_{n}}\} \ \wedge \ (u, v) \in E(G)\}| \text{,} \]
	gdzie \(\sigma \in S_{n}\) jest dowolną permutacją. Celem jest znalezienie
	permutacji \(\sigma \in S_{n}\), dla której wartość
	\[ \underset{i \in \{1, \ldots, n - 1\}}{\max} \
	\text{cutwidth}_{\sigma}(i) \]
	jest najmniejsza możliwa. Konkretnie, chcemy obliczyć tę wartość.
	
	\medskip
	
	Zdefiniujmy funkcję
	\[ \text{out}(X) \ = \ |\{(u, v) \ : \ u \in X \ \wedge \ v \in V(G)
	\setminus X \ \wedge \ (u, v) \in E(G)\}| \text{,} \]
	gdzie \(X\) jest dowolnym podzbiorem \(V(G)\). Wówczas
	\[ \text{cutwidth}_{\sigma}(i) \ = \ \text{out}(\{v_{\sigma_{1}}, \ldots,
	v_{\sigma_{i}}\}) \text{.} \]
	Oczywiście, dla konkretnego \(X\), wartość \(\text{out}(X)\) można łatwo
	obliczyć w czasie \(n^{\mathcal{O}(1)}\).
	
	\medskip
	
	Pozwala nam to skorzystać z programowania dynamicznego po podzbiorach. Niech
	\[ \text{dp}(X) \ = \ \min \left\{ \underset{i \in
	\{1, \ldots, |X| - 1\}}{\max} \ \text{cutwidth}_{\sigma}(i) \ : \ \sigma \in
	S_{n} \ \wedge \ \{v_{\sigma_{1}}, \ldots, v_{\sigma_{|X|}}\} = X \right\}
	\text{,} \]
	gdzie \(X\) jest dowolnym podzbiorem \(V(G)\). Wtedy
	\begin{align*}
		\text{dp}(\emptyset) \ &= \ 0 \text{,} \\
		\text{dp}(X) \ &= \ \min \{\max(\text{dp}(X \setminus \{x\}),
		\text{out}(X \setminus \{x\}) \ : \ x \in X\} \text{.}
	\end{align*}
	Naturalnie, wynikiem jest \(\text{dp}(\{1, \ldots, n\})\). Żeby obliczyć
	wartości \(\text{dp}\) dla wszystkich podzbiorów \(V(G)\), można na przykład
	przeglądać je w kolejności niemalejących rozmiarów. Całkowita złożoność
	naszego algorytmu wynosi \(2^{n} \cdot n^{\mathcal{O}(1)}\).
\end{document}
